\section{Introduction}

In the age of online platforms for entertainment, shopping and 
knowledge transfer, the automatic characterization of the available items and
the involved users has become a
crucial factor of success. 
% It is used to increase the ease of use 
% (bypass exhaustive searching of the database by suggesting appropriate pages),
% to maximize the effect of the platform (by placing appropriate recommendations,
% a user might buy/watch/read more than he originally intended) and to increase
% profit by intelligent ad placement and user screening for personalized marketing.
The tools created for such purposes are commonly called "recommender
systems".
In some cases, the recommendations can be done using additionally provided
information on users and items, such as demographic measures like age and gender or official
popularity measures and item classifications. This content-based approach tends
to be too simplistic and cumbersome for large databases and adds a dependeny
on third-party data of unknown accuracy.  
Methods grouped under the term "collaborative filtering" overcome these
difficulties by predicting behaviour based on past feedback. Feedback can either be
of implicit form e.g. how many times did a user click on an item, how long did he
watch a TV series, etc., or of explicit form, such as rankings or feedback form
surveys \cite{Hu2008}.
When users and items are grouped by their similarity and recommendations are
done based on similar users (user-oriented models) or similar items
(item-oriented models) respectively, one talks about neighborhood models. 
If, on the other hand, the items and users are characterized by a set of $K$ features
whose signification is a priori unknown, one talks about latent factor models.


The goal of this project is to apply collaborative filtering to the Netflix
dataset, which consists of explicit feedback in terms of ratings of $D=1000$ users
and $N=10000$ items. The main focus is on latent factor models, but neighborhood
models can be taken into account to improve predictions.
