\subsection{Ridge Regression}

\begin{figure}[htbp]
  \centering
  \includegraphics[width=.7\columnwidth]{figures/rmse_degree9final.png}
  \vspace{-3mm}
  \caption{rmse}
  \label{fig:rmse9}
\end{figure}

\begin{figure}[htbp]
  \centering
  \includegraphics[width=.7\columnwidth]{figures/cle_degree9final.png}
  \vspace{-3mm}
  \caption{cle}
  \label{fig:cle9}
\end{figure}

Cross valdiation for different polynomial degrees shows that a polynomial degree
higher than 7 was required to get predictions better than 80\%. 
The best results were obtained with degree 9, for which the predicted errors are
shown in Figures \ref{fig:rmse9} and \ref{fig:cle9}. For higher degrees, the
improvement in performance was not considered significant compared to the
compuational cost added. One can see that both the training and testing rmse
errors stagnate for small values of lambda, showing that this degree is not able to
perfectly capture the data. A similar trend can be seen in the classification
error, however the minimal test error occurs at a slightly higher value for
$\lambda$.  

The obtained weights are shown in Figure \ref{fig:weights}. It is interesting to
note that the first feature obtains a lot of weight, which confirms confirms the intuition that it is a good predictor.

\begin{figure}[htbp]
  \centering
  \includegraphics[width=.7\columnwidth]{figures/colormap29x9.png}
  \vspace{-3mm}
  \caption{Heatmap of optimum weight vector obtained by ridge regression.}
  \label{fig:weights}
\end{figure}

The predicted result on the training set and the result on the test set (online)
are shown in Table \ref{tab:ridge_regression}. 

\begin{table*}[htbp]
  \centering
  \begin{tabular}[c]{r|l||l|l|l|}
    \hline
    Degree & $\lambda$ & Prediction & Result  \\
    \hline
    7       & 10      &80.88     &80.57 \\ 
    7       & 1       &81.72       &81.70 \\ 
    9       & 0.21544 &81.94       &81.94 \\ 
    \hline
  \end{tabular}
  \caption{Final results ridge regression}
  \label{tab:ridge_regression}
\end{table*}

Results of the ridge regression

