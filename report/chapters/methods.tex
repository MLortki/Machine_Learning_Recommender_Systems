\section{Models and Methods}

\subsection{Preprocessing}

\subsubsection{Data analysis}

The data provided is of the form 
\begin{equation}
  \fat{X} = (x_{nd}),
\end{equation}
where $x_{nd}$ corresponds to the rating of user $d$ for movie $n$.  
the ratings are discrete and lie between 1 and 5. Since every user only rates a
very small subset of movies, the matrix is sparse. The data has the following
characteristics.

\begin{itemize}
  \item The trainig set provided has 991561 non-zero entries, denoted by $R(u,i)$ which corresponds to 
a fraction of $\approx 0.1$ with respect to the number of unknowns ($N \times $D). 
  \item Each user in the dataset has rated at least 8 movies, the maximum
    ratings per user is 4590. On average, a user rates roughly 1177 movies.
  \item Each movie in the dataset is rated by at least 3 users, and the maximum
    of ratings per movie is 522. On average, each move is rated by around 118
    users. 
\end{itemize}

\subsubsection{Splitting}
\label{sec:splitting}

The provided training data has been split into a training set and 2 small 
validation sets, as described more in detail in \ref{sec:blending}. k-fold cross-validation for was performed on the training set.

\subsubsection{Bias correction}
\label{sec:biascorrection}

Since users and movies can vary a lot in terms of their average ratings, 
there might be some implicit bias in
the data provided. 
This biais was removed as follows by subtracting a correcting term from each
element,
\begin{align}
  \widetilde{x}_{nd} &= x_{nd}-\mu_{nd} \\
  \mu_{nd} &=
 \begin{cases}
  \mu_{n} = \inv{|R(n)|} \sum_{d \in R(n)} x_{nd}, &\text{for item bias only}   \\
\mu_{d} = \inv{|R(d)|} \sum_{n \in R(d)} x_{nd}, &\text{for user bias only} \\
\mu = \inv{|R(n,d)|}\sum_{n,d \in R(n,d)} x_{nd}, &\text{for global bias only}
   \\
  \mu_{n} + \mu_{d} - \mu &\text{for combined biais}
 \end{cases}, 
  \label{eq:biases}
\end{align}

where the last term corresponds to what is used in \cite{Koren2009}. 
The entries of the residual matrix, $\widetilde{x}_{nd}$ are then factorized as explained
in \ref{sec:methods} and the bias is added again for the final predictions
(Figure \ref{fig:biasmatrix}).

\begin{figure}[htbp]
  \centering
  \includegraphics[width=.7\columnwidth]{figures/biases_user.png}
  \caption{A simple example for bias correction. The bias matrix is obtained by
  substracting the mean of the corresponding user. The matrix composed of these
  means and the biases sum up to the original training matrix as expected.}
  \label{fig:biasmatrix}
\end{figure}

The performance with and without corrections was tested on ALS (Figure
\ref{fig:bias}. As expected, the combined bias with underlying assumption that each rating can
be composed of a contribution by the user and one by the item, is the most
accurate model. 

\begin{figure}[htbp]
  \centering
  \includegraphics[width=.7\columnwidth]{figures/bias.jpg}
  \vspace{-3mm}
  \caption{Performance of ALS algorithm using different variants of bias
  correction as shown in \eqref{eq:biases}}
  \label{fig:bias}
\end{figure}


\subsection{Machine learning methods}
\label{sec:methods}

\subsubsection{Library Review}

The surprise library was chosen
for its high performance thanks to customizability, great ease of startup
 (installation \& system requriements), ease of use and big variety of
 implemented algorithms for recommender systems.

 TODO: insert links/references
\begin{table}
  \centering
\begin{tabular}{|l|c|c|c|c|}
  \hline
   & startup & ease of use & performance & variety \\
  \hline 
  graphlab   & \two & \thr & \one & \two \\
  pyspark     & \one & \one & \two & \thr \\
  surprise    & \thr & \thr & \thr & \two \\
  fancyinput  & \thr & \thr & \two & \one \\
  \hline 
\end{tabular}
  \caption{Overview of libraries tested for recommender system implementation.
  One star corresponds to lowest performance. }
  \label{tab:libraries}
\end{table}

The goal is to factorize the given ratings matrix using two low-rank matrices, 
\begin{equation}
  \fat{X} = \fat{W}\fat{Z}^T \with \fat{W} \inR{N \times K},
  \fat{Z} \inR{D \times K}, 
\end{equation}
where $K$ is the number of latent features, and \fat{Z} and \fat{W} are in the following
referred to as the user and feature matrix respectively.

Find more about these methods in \cite{Aberger2009}

\subsubsection{Bias Stochastic Gradient Descent}

The stochastic gradient descent (SGD) algorithm used in this report is provided by the
\textit{Surprise} library
\footnote{\href{}{https://github.com/NicolasHug/Surprise}} by Nicolas Hug.  
It is based on the algorithm provided by  
Simon Funk \footnote{\href{}{http://sifter.org/~simon/journal/20061211.html}}, 
which uses bias correction as described in \cite{Koren2009}. 
Without this bias correction, it reduces to the algorithm given in \cite{Salak2008}. 

As shown in \ref{sec:biascorrection} and by \cite{Aberger}, we expect better predictions when removing
the user and item biases, so we choose to use this correction and we choose the
other hyperparameters using cross-validation. 





\subsubsection{Alternating Least Squares}

Regularization added from \cite{Zhou2008}


\subsection{Blending}

It has been shown that if one disposes of several well performing methods,
a clever combination of the methods can improve the accuracy of predictions.   
A combination can overcome problems when the used algorithms are prone to
converge to local optima, or full convergence is too computationally expensive.
\cite{Dietterich}
A combination of multiple algorithms can be particularly beneficial when some
are shown to perform better in certain regions of the features space than
others, giving each method a higher weight in the regions where it performs
best.

The splitting of the kaggle dataset is done as suggested in
\cite{Andreas2009}: each method is trained on a fixed subset of 95\% of the
dataset (training set), while the remaining 5\% are split up equally, and one half is used to
compare the different methods (probe set) and the other half is used to create a
valuable prediction of the performance of the final, blended method (test set).  
This splitting ensures at least 30'000 non-zero entries in the smallest probe
and test sets,
which was considered enough for the evaluation, and removes only a small
proportion of the valuable training data.

The process of linear blending is shown in Figure \ref{fig:blending}. 
The blending model is obtained by minimizing the mean squared prediction error
on the probe set as suggested in \cite{Andreas2009}. 
If we call $\mathbf{p}_i$ the vector of $N_P$ ordered predictions of
method $i$ ($i = 1\ldots N_M$) on the
probe set, and $\mathbf{r}$ the true values (in the same order), then the weight vector
$\mathbf{x}$ is obtained by solving 

\begin{align}
  \mathbf{P} &=[\mathbf{p}_0, \mathbf{p}_1, \ldots, \mathbf{p}_{N_M}] \inR{N_P\times N_M}, \\
  \mathbf{x} &= (\mathbf{P}^T\mathbf{P})^{-1}\mathbf{P}^T\mathbf{r} \with
  \mathbf{x},\mathbf{r}\inR{N_M}.
\end{align}

The best prediction is obtained by applying these weights to the predictions of
the methods on the kaggle set of size $N_T$, $\mathbf{q}_i$, yielding the optimum ratings
vector,

\begin{align}
  \mathbf{Q} &=[\mathbf{q}_0, \mathbf{q}_1, \ldots, \mathbf{q}_{N_M}]
  \inR{N_T\times N_M}, \\
  \hat{\mathbf{q}}_i &= \mathbf{Q}\mathbf{x} \inR{N_T}.
\end{align}

\begin{figure}[htbp]
  \centering
  \includegraphics[width=\columnwidth]{figures/blending2.png}
  \vspace{-3mm}
  \caption{Visualization of blending process used to combine the
  methods (\textbf{orange}: datasets, 
  \textbf{green}: operations, \textbf{blue}: results).
  The models are trained on a subset of the kaggle training set and 
  predictions tested on the probe set to evaluate their performance 
  ($p_i$ vs. $r$). The predictions ($q_i$) for the kaggle test set
  are optimally weighted using linear regression, and the obtained blending method is tested on the test
  to predict its performance.}
  \label{fig:blending}
\end{figure}


\subsection{Evaluation}

\begin{enumerate}
  \item TODO: write about why kaggle results tend to be different than lcoal data for
    SGD, but not for ALS.
  \item \textbf{Cross valdiation} A 10-fold cross validation is implemented
    for the whole training dataset to fix the hyperparameters. 

  \item \textbf{Performance prediction} The training data was split into two
    sets of ratio 1:2, emulating the actual data ratio between Kaggle's training
    and test set. Doing the matrix factorization with this training/test set
    pair enables us to predict if we can expect a better performance on the
    kaggle dataset.  
    TODO: what is this ratio for this dataset? 
\end{enumerate}

